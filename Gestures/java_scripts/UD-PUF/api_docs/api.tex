%% The comment character in TeX / LaTeX is the percent character.
%% The following chunk is called the header

\documentclass{article} % essential first line
\usepackage{times}    % this uses fonts which will look nice in PDF format
\usepackage{graphicx}   % needed for the figures
\usepackage{url}
\usepackage{adjustbox}
\usepackage{amsmath}
\usepackage{listings}
\usepackage{color}
\usepackage{float}

\definecolor{dkgreen}{rgb}{0,0.6,0}
\definecolor{gray}{rgb}{0.5,0.5,0.5}
\definecolor{mauve}{rgb}{0.58,0,0.82}

\lstset{frame=tb,
  language=Java,
  aboveskip=3mm,
  belowskip=3mm,
  showstringspaces=false,
  columns=flexible,
  basicstyle={\small\ttfamily},
  numbers=none,
  numberstyle=\tiny\color{gray},
  keywordstyle=\color{blue},
  commentstyle=\color{dkgreen},
  stringstyle=\color{mauve},
  breaklines=true,
  breakatwhitespace=true,
  tabsize=3
}

%% Set the folder where pictures are located
\graphicspath{ {images/} }

%% Here I adjust the margins

\oddsidemargin -0.25in    % Left margin is 1in + this value
\textwidth 6.75in   % Right margin is not set explicitly
\topmargin 0in      % Top margin is 1in + this value
\textheight 9in     % Bottom margin is not set explicitly
\columnsep 0.25in   % separation between columns

%% Define a macro for inserting postscript images
%% ==============================================
%% This is a macro which nominally takes 3 parameters, 
%% it would be used as follows to insert and encapsulated postscript
%% image at the location where it is used.
%%
%% \EPSFIG{epsfilename}{caption}{label}
%% - epsfilename is the name of the encapsulated postscript file to be
%%               inserted at this location
%% - caption is the text to be shown as the figure caption, it will be
%%           prepended by Figure X.  The number X can be referenced
%%           using the label parameter.
%% - label is a name given to the figure, it can be referenced using the
%%         \ref{label} command.

%\def\EPSFIG[#1]#2#3#4{   % Don't be scared by this monsrosity
%\begin{figure}[hbt]    % it is a macro to save typing later
%\begin{center}     % 
%\includegraphics[#1]{#2} %
%\end{center}     %
%\caption{#3}     %
%\label{#4}     %
%\end{figure}     %
%}        %

%% Define the fields to be displayed by a \maketitle command
\author{Timothy Dee, Nicholas Montelibano}
%{\it Undergraduate, Department of Electrical and Computer Engineering, Iowa State University}
\title{UD-PUF API}

%%
%% Header now finished
%%
\begin{document}    % Critical
%\twocolumn
\thispagestyle{empty}   % Inhibit the page number on this page
\maketitle      % Use the \author, \title and \date info

% List out and describe the public API
\section{Public API}
\subsection{UserDevicePair}
This class is the most useful object in the UD-PUF library. It is used to represent a user device combination. This class implements the authentication component of the library.
This class holds a list of challenges for a given user. For each challenge there are presumably multiple responses.
Authentication entails comparing a new response for a given challenge to the existing profile built for that challenge.
\subsubsection{Constructors}
Below are listed the constructors for the UserDevicePair class. The final constructor on line 5 provides the ability to create a UserDevicePair object by specifying all of the parameters.
The other constructors provide default parameters for the parameters which are not specified.

\begin{table}[H]
\centering
\begin{tabular}{ | l | l | p{6.5cm} | p{3cm}|}
\hline

Type & Parameter & Meaning & Default Value \\ \hline
int & userDeviceID & UserDeviceID can be used to keep track of the user & \\ \hline
List\textless Challenge\textgreater & challenges & A list of Challenge objects corresponding to challenges already constructed for the user, device & empty ArrayList\textless Challenge\textgreater \\ \hline
double & allowed\_deviations & This is a parameter used during authentication. When testing for points which don't match the profile this value determines the number of standard deviations a point in a response may be away from the corresponding average for that point in the profile. 
Points which fall outside of this number of standard deviations will be considered not within the profile.
We call these failed points. & 1.0 \\ \hline
double & authentication\_threshold & The ratio of failed points to total points at which the user will be considered within the profile and pass authentication. & 0.75 \\ \hline

\hline
\end{tabular}
\end{table}

\begin{lstlisting}[numbers=left]
public UserDevicePair(int userDeviceID){}

public UserDevicePair(int userDeviceID, List<Challenge> challenges){}

public UserDevicePair(int userDeviceID, List<Challenge> challenges, double allowed_deviations, double authentication_threshold){}
\end{lstlisting}

\subsubsection{Public Methods}
\begin{lstlisting}[numbers=left]
// Adds challenge to list of challenges correlating to this user/device pair
public void addChallenge(Challenge challenge){}

// gets the challenges for this user, device
public List<Challenge> getChallenges(){}

/**
 * true if the new_response_data has a certain percentage of points which
 * fall within the profile for the challenge indicated by challenge_id
 * 
 * The testPressListVsDistrib() method in the Util file seems to be
 * performing the authentication
 */
public boolean authenticate(List<Point> new_response_data, int challenge_id){}

// return the userDeviceID
public int getUserDeviceId(){}

/**
 * return the number of failed points from the previous authentication.
 * Return -1 if there is not previous authentication.
 */
public double failedPointRatio(){}
\end{lstlisting}

\subsection{Profile}
\subsubsection{Constructors}
%TODO
\begin{lstlisting}[numbers=left]
%TODO
\end{lstlisting}

\subsubsection{Public Methods}
%TODO
\begin{lstlisting}[numbers=left]
%TODO
\end{lstlisting}

\subsection{Response}
\subsubsection{Constructors}
Response constructor takes in a responsePattern. This responsePattern represents the Points the user traced in response to the provided challenge.
\begin{lstlisting}[numbers=left]
public Response(List<Point> responsePattern){}
\end{lstlisting}

\subsubsection{Public Methods}
Normalize method preforms the following function. Normalizes points in response. The normalizingPoints are a list of points to normalize the response to. In other words the response will then contain exactly these point having some pressure determined by the original response.
\begin{lstlisting}[numbers=left]
public List<Point> getResponse() {}

public void normalize(List<Point> normalizingPoints, boolean isChallengeHorizontal) {}
\end{lstlisting}

\subsection{Challenge}
\subsubsection{Constructors}
Takes a list of Points corresponding to the challenge points presented to the user.
\begin{lstlisting}[numbers=left]
public Challenge(List<Point> challengePattern, int challengeID) {}
\end{lstlisting}

\subsubsection{Public Methods}
The challenge contains a number of responses. These responses correspond to the responses generated by the user when they are presented this challenge. A response is normalized when it is added to the Challenge.
\begin{lstlisting}[numbers=left]
// add a response to this challenge
// this method will normalize the response before adding it
public void addResponse(Response response) {}

// return the mu sigma profile for the responses to this challenge
public Profile getProfile() {}

// return the challenge points
public List<Point> getChallengePattern() {}

// return the ID of this challenge
public int getChallengeID() {}

// Determine if the challenge is more horizontal than vertical in oreantation
public boolean isHorizontal(){}
\end{lstlisting}

\subsection{Point}
\subsubsection{Constructors}
Constructor which take x, y, pressure represented x position, y position, and pressure of the point respectively.
\begin{lstlisting}[numbers=left]
public Point(double x, double y, double pressure) {}

public Point(Point p) {}
\end{lstlisting}

\subsubsection{Public Methods}
\begin{lstlisting}[numbers=left]
public double getX() {}

public double getY() {}

public double getPressure() {}
    
// compares each of x, y, pressure for equality
public boolean equals(Object p) {}
\end{lstlisting}

% example use of the library
\section{Examples}
% describes the process to create and use a UserDevicePair
\subsection{Creating a UserDevicePair}

\begin{lstlisting}[numbers=left]
  Challenge challenge;
  Response response;
  List<Point> response_points;

  // create a userDeficePair
  ud_pair = new UserDevicePair(0);

  // create a list of challenge points
  List<Point> challenge_points = new ArrayList<Point>();

  // sample points for testing
  challenge_points.add(new Point(100, 100, 0));
  challenge_points.add(new Point(200, 100, 0));
  challenge_points.add(new Point(300, 100, 0));
  challenge_points.add(new Point(400, 100, 0));

  // add the challenge to it which I want to authenticate against
  // create 3 responses to add to this challenge
  challenge = new Challenge(challenge_points, 0);

  for (int i = 0; i < 3; i++) {
      response_points = new ArrayList<Point>();

      // create the response
      for (int j = 0; j < 32; j++) {
    response_points.add(new Point((300 / 32) * j + 100, 100, i));
      }

      response = new Response(response_points);
      challenge.addResponse(response);
  }

  // the mu sigma for the responses should be
  // mu : 1
  // sigma : sqrt(2/3)
  ud_pair.addChallenge(challenge);
\end{lstlisting}

\subsection{Creating a Response Object}
\begin{lstlisting}[numbers=left]
  // create the response object
  List<Point> response_points = new ArrayList<Point>();

  // populate the response_points list with 10 points
  for (int i = 0; i < 9; i++) {
      response_points.add(new Point(i, i, .1 * i));
  }

  response = new Response(response_points);
\end{lstlisting}

\subsection{Creating a Challenge Object}
\begin{lstlisting}[numbers=left]
  // construct some test data
  List<Challenge> challenges = new ArrayList<Challenge>();
  List<Point> challenge_points = new ArrayList<Point>();

  // sample points
  challenge_points.add(new Point(100, 100, 0));
  challenge_points.add(new Point(200, 200, 0));
  challenge_points.add(new Point(300, 300, 0));
  challenge_points.add(new Point(400, 400, 0));

  challenges.add(new Challenge(challenge_points, 0));
\end{lstlisting}

\end{document}