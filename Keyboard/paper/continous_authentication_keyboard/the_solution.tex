% idea: a re-usable insight, useful to the reader
%   one clear, sharp idea
\section{The Solution}
\label{the_solution}
% explain the main idea
The main idea of this paper is that user
%
touch screen interactions may be used in order to
distinguish a legitimate user of a mobile device apart from
illegitimate users.
%
We implement a continuous authentication system
based on user interactions with the soft keyboard of
a mobile device.
This system uses properties of the interactions including
pressure and location.

% explain that the user of the soft keyboard generates necessary sequence of data
A large part of a user's interactions with a mobile
device involve the input of data with a soft keyboard.
These soft keyboard applications require that
the user put their finger on the screen at a consistent
location to indicate a given letter should be taken as input.
This input is rich with information including
pressure,
key code, and
timestamp.
As the keys on the soft keyboard are always in the same place on the screen,
we take the key code value to represent the location in our model.
Through interactions with multiple applications over time,
the user produces a sequence of these inputs.
This sequence is used to construct a model of user behavior.

% say that this sequence is unique to a user and device
% explain what makes it unique to the user
% explain what makes it unique to the device
The goal in creating the model
is to be able to distinguish the behavior
of one user from 
another user on the same device and
from the same user on a different device.
That is, the pairing of user and device will
create a model unique to that pair.
If either the user or device is changed,
the model will be sufficiently different
that it is detectable.
The input sequence will be used to characterize the user-device pair.
%TODO cite properties of device silicon
This uniqueness is possible because
the pressure metric is a product of 
unique properties in
the silicon of the device and
the finger of the human user.
The silicon's unique properties
are a product of the fabrication process while %TODO \cite{}
the uniqueness of the human user is derived from
the way in which they touch the screen.

% motivate the choice of a Markov model though example
Let's say we want to model the behavior
of an individual in terms of where 
they choose to spend their time.
Say further that the goal in creating
this model is to predict their $t+1^{st}$ location
based on their current location.
%
The outcome of using a Markov model to describe a system
is a vector $\hat{P}$ of probabilities for each possible state.
For the individual in our behavior model,
$\hat{P}_i$ corresponds to the probability that the $i_{th}$ location
will be the location of the person at time $t+1$ if the current time is $t$.
%
Such a probability outcome for our individual might be
that if at time $t$ the person is in the living room,
then at time $t+1$ there might be 
a $70\%$ probability they are still in the living room,
a $20\%$ probability they are in the kitchen, and
a $10\%$ probability they are in the bathroom.
%
Such a model might be useful in understanding 
the behavior patterns of one person, or
comparing behavior patterns among persons.

% explain what a full Markov model is
% be sure to explain tokens, window, model building process, what outcome probabilities mean
If a Markov model can describe the trans
%TODO

% explain how the n-gram model differs from the full Markov model
% explain the n in n gram
% articulate the speed advantage (and other advantages) for using an n-gram
% articulate the disadvantage in precision/ accuracy
%TODO

% describe the n-gram model we use
% what does n mean in this context?
%TODO

% what are tokens in our model?
Our $n$-Markov model uses tokens
which are a tuple of location and pressure
generated through user touch screen interactions. 
There are a very large number of possible
location, pressure combinations.
Since it is unlikely that the user will be
very precise in location or pressure,
having such a large number of tokens
creates a situation where each sequence of 
touch interactions will be different
for the same user on the same device;
this is not desirable.
In fact, if the entire space were used 
the variations in location and pressure,
even when generated by a single user,
would result in a very large number of tokens
each having low probabilities of succeeding any
$n$ token sequence.

% describe how the space is divided up
% how are tokens chosen
We require some way of splicing this space
%TODO

% explain how this model is used to compute probabilities (in general)
Let us return to the situation where
a Markov model was used to predict the $t+1^{st}$ location
of an individual.
Now suppose that these probabilities have been developed for
two people who's behavior patterns we would like to
compare.
Our goal is to quantify the difference between
the location patterns of these individuals.
%TODO describe how to compare these location patterns (develop the probabilities and mathematically find the difference) 
%TODO

% describe the specifics of the probability computation
%TODO

% describe what is a prefix tree
% say WHAT it is used to store
%TODO

% describe HOW things are stored in the prefix tree
% answer why this is done
% discuss why this is useful (perhaps with math)
%TODO

% finally, relate this model back to the claims
% describe how this is the solution
%TODO