% here the evidence that the claims are correct is presented
% the structure, about 3 paragraphs for each analysis
% 1 This is my analysis
% 2 This analysis shows/demonstrates x
% 3 x supports my claim in the following way
\section{The Details}
\label{the_details}

%TODO move these figures to where they need to go in the paper
\ExecuteMetaData[graphics.tex]{normal}
\ExecuteMetaData[graphics.tex]{qq}
\ExecuteMetaData[graphics.tex]{window}
\ExecuteMetaData[graphics.tex]{authentication}
\ExecuteMetaData[graphics.tex]{total}
\ExecuteMetaData[graphics.tex]{nexus}

% begin by describing what our implementation is
% answer the question, "what did you do"
In order to answer the question of
whether or not touch screen interactions may be used
in order to distinguish between users
a system is implemented to gather and analyze data
from interactions with the touchscreen of an Android device.
This data consists of location, pressure, and time
values associated with user's interactions with a soft keyboard.
%
For gathering purposes,
a keyboard application was modified to record
the necessary data values.
%
Two users were enlisted to acquire a large 
amount of data on two different Nexus 7 tablets.
In effect this creates four user, device combinations.
%
In order to analyze the collected data,
the system described in \ref{the_solution}
has been constructed.

% discuss false_positive and false_negative percentages
% what are these metrics (definitions)
% what did we find them to be (values)
In order to measure the performance of the system
it is useful to define some metrics
which give insight into the user experience.
%
We use false positive percentage,
the ratio of identities which are incorrectly classified as the user
to the total number of identities authenticated,
and
false negative percentage,
the ratio of identities which are incorrectly classified as not the user,
to the total number of identities authenticated.
%
These metrics can be used to describe how the device will
behave from the user's perspective.
%
False positives allow illegitimate users access to the device
which may result in loss of data should the device be compromised.
%
On the other hand,
false negatives deny device access to the legitimate user.
A large number of false negatives would conceivably 
be very frustrating to the user.
%
There is a natural inverse relationship between
false positive percentage and false negative percentage.
The authentication can be made less restrictive to 
engender fewer false negatives, but
in exchange a greater proportion of
illegitimate users will also authenticate.
%
This relationship is exhibited in Figure \ref{fig:threshold_vs_percentages}
where one of our system's parameters is adjusted to 
allow varying rates 
of false positive percentage and false negative percentage.

% picture of false positives and false negatives in our system
\ExecuteMetaData[graphics.tex]{threshold}

% discuss authentication_accuracy
% what is is this metric
% how is it measure
% why is it a good way to measure how well the system preforms
Another metric used to describe the performance of our system
is authentication accuracy.
%
Defined as the 
number of correct authentications by
the total number of authentications,
the goal of this metric is to
aggregate
false positive percentage and
false negative percentage 
to say something about how well our system can distinguish between
legitimate and illegitimate identities.
%
Authentication accuracy is 
expressed as a percentage
related to 
false positive percentage and
false negative percentage by
$authentication_accuracy = 1.0 - (false_positive_percent) - (false_negative_percent)$.
%
This is intuitive as both 
false positive percentage and
false negative percentage 
represent incorrect authentication results.

% discuss our authentication accuracy findings
% how well does the system do overall
%TODO

% discussion of the model parameters in general
% say what each of them were and their impact on the system
% window, token, time_threshold, (user_model_size, auth_model_size), authentication_threshold
%TODO

% now go into detail on each of the model parameters affected the system
% present data with regards to each of the parameters
% 1 This is my analysis
% 2 This analysis shows/demonstrates x
% 3 x supports my claim in the following way
%
% window
%TODO

% token
%TODO

% time_threshold
%TODO

% user_model_size and auth_model size
%TODO

% authentication threshold
%TODO

% discuss iterations of the difference computation
% say what worked and what didn't work
% discuss what we were seeing that motivated specific design choices
%
% weighted windows and tokens in difference computation
% why did we do this?
The windows and tokens in the \dots 
are weighted by \dots
to help \dots
%TODO

% define confidence interval for the difference computation
%TODO

% talk about how this correlates with the Jensen Shannon divergence between the sets
%TODO

% discuss computation time on the nexus 7
% How does changing different model parameters affect run time
%
% user_model_size and auth_model_size
%TODO

%TODO was there no significant difference for changing the other parameters?
%TODO

% describe the data collection process
%TODO

% describe how the data looked 
% (normal distribution?)
% (q,q plot?)
% how fine grained was the pressure data?
% what did this cause us to decide to do?
%TODO

% describe how the results were analyzed
% in other words what did we do to produces the numbers presented above
The data is analyzed by 
%TODO 

% include mention of some problems which presented themselves
Some notable problems presented themselves
throughout the course of this work
which could influence the usability of such a 
system in a practical sense.
%
The Android \textbf{MotionEvent} class \textbf{getPressure()} method
used to collect pressure data in our system
does not always return a high granularity of values.
The pressure values are sometimes grouped
into large steps.
There might, for example,
be $8$ steps between $0.0$ and $1.0$ on some devices.
This is a problem for our system which uses
these pressure values to understand user variability.
%
The problem seems to be device dependent
suggesting that the device drivers may have a role to play
in decreasing the reported graininess of pressure.
Another possibility might be the values
reported by the sensors,
Perhaps these sensors only report a small number of steps.

% state the performance impact on our system
% say how fine grained the pressure data is on the nexus 7
On the Nexus $7$ tablets used to test this implementation,
the pressure data from the touchscreen is 
reported at a precision of $0.096$ by the {\tt getPressure()} method.
This measurement has been determined
by finding the minimum difference between
any two pressure values in one of our data sets.
%
There is a small probability the precision could
be greater if two adjacent pressure measurements
were never taken in the dataset.
%
The precision of {\tt getPressure()} is
known to vary among devices.
We have not encountered any inconsistency
in the precision among devices of the same type (eg. Nexus $7$).
%
Devices with less precise values returned from
{\tt getPressure()}
compared to those pressure values on the Nexus $7$
tended to preform worse.
%
This makes intuitive sense as the basis for the system
is that pressure may be used as a user and device biometric.

% tie all analysis's back to the claims
Where does this leave us?
The goal in preforming these tests was to show it is possible to construct
an 
%
unobtrusive system which 
establishes a user identity 
and a device identity
based on pressure values 
derived from user touch screen interactions.
%
Further to show that this identity may be 
distinguished from other 
user and device identities with accuracy ($70-90+\%$)
in ($1-5 sec$) on a Nexus $7$ tablet.
%
% discuss how all of these claims have been accounted for
All of these claims have been accounted for
in the above discussion.
%
% relate back to solving the problem of 
% insufficient authentication after the lock screen has been bypassed
Such a system provides a reasonable step forward in
securing mobile devices in cases where the lock screen has
been bypassed by an illegitimate user.