% here the evidence that the claims are correct is presented
% the structure, about 3 paragraphs for each analysis
% 1 This is my analysis
% 2 This analysis shows/demonstrates x
% 3 x supports my claim in the following way
\section{The Details}
\label{the_details}

%TODO move these figures to where they need to go in the paper
\ExecuteMetaData[graphics.tex]{normal}
\ExecuteMetaData[graphics.tex]{qq}
\ExecuteMetaData[graphics.tex]{threshold}
\ExecuteMetaData[graphics.tex]{window}
\ExecuteMetaData[graphics.tex]{authentication}
\ExecuteMetaData[graphics.tex]{total}
\ExecuteMetaData[graphics.tex]{nexus}

% begin by describing what our implementation is
% answer the question, "what did you do"
In order to answer the question of
whether or not touch screen interactions may be used
in order to distinguish between users
a system is implemented to gather and analyze data
from interactions with the touchscreen of an Android device.
This data consists of location, pressure, and time
values associated with user's interactions with a soft keyboard.
%
For gathering purposes,
a keyboard application was modified to record
the necessary data values.
%
Two users were enlisted to acquire a large 
amount of data on two different Nexus 7 tablets.
In effect this creates four user, device combinations.
%
In order to analyze the collected data,
the system described in \ref{the_solution}
has been constructed.

% discuss false_positive and false_negative percentages
% what are these metrics (definitions)
% what did we find them to be (values)

% discuss authentication_accuracy
% what is is this metric
% how is it measure
% why is it a good way to measure how well the system preforms
%TODO

% discussion of the model parameters in general
% say what each of them were and their impact on the system
% window, token, time_threshold, (user_model_size, auth_model_size), authentication_threshold
%TODO

% now go into detail on each of the model parameters affected the system
% present data with regards to each of the parameters
% 1 This is my analysis
% 2 This analysis shows/demonstrates x
% 3 x supports my claim in the following way
%
% window
%TODO

% token
%TODO

% time_threshold
%TODO

% user_model_size and auth_model size
%TODO

% authentication threshold
%TODO

% discuss iterations of the difference computation
% say what worked and what didn't work
% discuss what we were seeing that motivated specific design choices
%
% weighted windows and tokens in difference computation
%TODO

% TODO something else we changed in the difference computation
%TODO

% discuss computation time on the nexus 7
% How does changing different model parameters affect run time
%
% user_model_size and auth_model_size
%TODO

%TODO was there no significant difference for changing the other parameters?
%TODO

% describe how the results were analyzed
% in other words what did we do to produces the numbers presented above
The data is analyzed by 
%TODO 

% include mention of some problems which presented themselves
Some notable problems presented themselves
throughout the course of this work
which could influence the usability of such a 
system in a practical sense.
%
The Android \textbf{MotionEvent} class \textbf{getPressure()} method
used to collect pressure data in our system
does not always return a high granularity of values.
The pressure values are sometimes grouped
into large steps.
There might, for example,
be $8$ steps between $0.0$ and $1.0$ on some devices.
This is a problem for our system which uses
these pressure values to understand user variability.
%
The problem seems to be device dependent
suggesting that the device drivers may have a role to play
in decreasing the reported graininess of pressure.
Another possibility might be the values
reported by the sensors,
Perhaps these sensors only report a small number of steps.
%TODO state the performance impact on our system

% tie all analysis's back to the claims
%TODO