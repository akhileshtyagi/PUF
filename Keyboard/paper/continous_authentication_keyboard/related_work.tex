% "this bit of work came from this guy, 'this' is what I added to it"
% phrase the things which were added in terms of BENEFITS and DRAWBACKS
\section{Related Work}
\label{related_work}
% mention ryan scheel's work on gestures authentication
In \cite{ScheelTyagi15}
a physical unclonable function (PUF) composed of
a human biometric and silicon biometric
leading to a unique user device identity.
This work demonstrates how the PUF may be used
for authentication.
%
The user is presented with a polyline draw on the touch screen.
The user then traces this line.
The human pressure exerted in the trace and
the exact traced path profile captures
a biometric of the user.
%
The silicon variability is extracted
in a similar way to our implementation
through capacitive touch and sensor circuitry
of the mobile device.
%
% how does our work differ
Our work differs in that
we derive a model of user behavior over time
which describes how the user
interacts with the soft keyboard.
Comparatively \cite{ScheelTyagi15} shows
that, once trained for a particular challenge line,
it is possible to differentiate pair of user, device from
other pairs of user, device.
%
In addition,
we use discrete touchscreen events, key presses, while 
\cite{ScheelTyagi15} uses continuous touchscreen events, swipes.

% need to describe the Mellon Sensec paper
More good work done in the area of
PUF's on mobile devices includes \cite{sensec13}.
%
This work utilizes the silicon manufacture variability 
found in many different sensors
combining them to create a model of the user.
Again, the variability for this system also
utilizes human biometrics.
%
\cite{sensec13} uses data from
the accelerometers, gyroscopes, and magnetometers
to develop their model.
By comparison,
our system shows this may be done for touch screen interactions.
%
%TODO describe how the modeling idea (MLE) came from sensec
%TODO
%
%TODO describe our advantage over sensec
%Our advantage over this work is....

%TODO look for other work characterizing user using touchscreen interactions