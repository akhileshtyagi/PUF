% "this bit of work came from this guy, 'this' is what I added to it"
% phrase the things which were added in terms of BENEFITS and DRAWBACKS
\section{Related Work}
\label{related_work}
% from conference version of paper
%%%
% compare our work to other biometric work
There is significant amount of work on PUFs \cite{Devadas:2009:PUF}, \cite{PUFIntro}, \cite{Gassend:2002:SPR}
Ratha et al. \cite{Ratha:2001} proposed use of sensor biometrics for authentication.
%
In the continuous authentication world, a composite sensor vector \cite{zhu2013sensec} has been used to establish a user identity. 
This system, Sensec, builds the composite sensor
utilizing data from accelerometers, gyroscopes and magnetometers.
%
Other user biometrics such as inter key stroke timing 
\cite{KeystrokeHASE14},  accelerometer based signature \cite{Liu:2009:UAP}, MEMS sensors
\cite{Aysu:2013:DFL}, Gait sensors \cite{Gait}, gestures \cite{touchscreengestures}, \cite{Gesture14}, and 
broader sensor set \cite{Dey:2013:AHP} have also been
used. 
SenGuard \cite{shi2011senguard} is a similar to SenSec in creating a passive
authentication mechanism using sensor based models. In this paper, the continuous authentication is based on a combined user-device identity,
which is derived from a user-device (UD)-PUF \cite{ScheelTyagi15}.
%
\cite{ruhrmair2015virtual} presents the idea of
"virtual proofs of reality" 
also referred to as "virtual proofs" (VP).
The aim of this work on VPs is
to prove physical statements over digital communication channels.
%
Our work can be seen as
a VP of the user's presence at the touchscreen.

% citation mentioned in HOST review
Other works \cite{rosenfeld2010sensor} have proposed 
entangling physical measurements with sensor data
to produce PUFs.
%
The goal of \cite{rosenfeld2010sensor} is to 
extend the functionality of conventional PUFs
to provide authentication, unclonability, and verification of sensed values.
\cite{rosenfeld2010sensor} proposes the integration of PUFs into sensors
at the level of hardware.
These hardware PUF sensors use sensed values as part of the challenge,
generating a different response depending on the physical measurement.
%
The work in this paper uses software to exhibit
the capacity of standard touchscreen sensors to provide PUF functionality.
%%%
% end from conference version

% mention ryan scheel's work on gestures authentication
In \cite{ScheelTyagi15}
a physical unclonable function (PUF) composed of
a human biometric and silicon biometric
leading to a unique user device identity.
This work demonstrates how the PUF may be used
for authentication.
%
The user is presented with a polyline draw on the touch screen.
The user then traces this line.
The human pressure exerted in the trace and
the exact traced path profile captures
a biometric of the user.
%
The silicon variability is extracted
in a similar way to our implementation
through capacitive touch and sensor circuitry
of the mobile device.
%
% how does our work differ
Our work differs in that
we derive a model of user behavior over time
which describes how the user
interacts with the soft keyboard.
Comparatively \cite{ScheelTyagi15} shows
that, once trained for a particular challenge line,
it is possible to differentiate pair of user, device from
other pairs of user, device.
%
In addition,
we use discrete touchscreen events, key presses, while 
\cite{ScheelTyagi15} uses continuous touchscreen events, swipes.

% need to describe the Mellon Sensec paper
More good work done in the area of
PUF's on mobile devices includes Sensec \cite{sensec13}.
%
This work utilizes the silicon manufacture variability 
found in many different sensors
combining them to create a user identity.
Again, the variability for this system  is also
derived from human biometrics.
%
Sensec uses data from
the accelerometers, gyroscopes, and magnetometers
to develop their model.
In this work,
raw data is converted into a textual representation.
An $n-gram$ Markov model is 
then trained with this textual representation.
%
Our system also utilizes a similar $n-gram$ Markov model, but
our model allows us to say significantly different
things about the user
because it is tailored toward aspects of 
a single physical event, user touch screen interactions.
%
With our model,
it is straightforward to make statements about the user behavior.
We can say the user has some $\%$ likelihood of pressing a key
with a certain pressure given the previous few touch screen interactions.
%
No such deduction can be made in an obvious way from the Sensec model.
%
%Because our model is tied physically
%
Sensec requires information from an array of sensors
while
our system shows pressure from touch screen interactions is
a sufficient user biometric to classify users with high accuracy.
%
%TODO describe how the modeling idea (MLE) came from sensec
%TODO
%
%TODO describe our advantage over sensec
%Our advantage over this work is....
