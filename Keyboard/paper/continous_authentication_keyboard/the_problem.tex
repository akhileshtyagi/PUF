% here is the problem
% it is an interesting problem
% NOTE: remember to introduce problems using examples
% 1: present example
% 2: describe problems in example
% 3: present the general case of problems
\section{The Problem}
\label{the_problem}
% what are the issues ( restate from introduction )
Current protection mechanisms do not adequately protect
data contained on mobile devices.
The current scheme provides a single line of defense,
the lock screen.
If this line of defense has been compromised
by an attacker,
all applications and data on the device become available.
%
This scheme might be considered successful 
if the lock screen offered a high level of device security.
Unfortunately, there are exploits for many common lock screen 
authentication mechanisms 
allowing an attacker to bypass the protection.
%
Common protection mechanisms include
pin, password, pattern, finger print scanners, and facial recognition.
%
The first three can be overcome successfully with 
shoulder-surfing \cite{schaub2012password} and smudge attacks \cite{aviv2010smudge}.
Finger print scanners and facial recognition require a different approach
where the relevant features of the legitimate user are spoofed
in order to pass authentication.
%
A system is created in \cite{cao2016hacking} 
to easily and quickly print finger prints
capable of passing finger print scanner authentication.
%
Similarly \cite{de2013can} describes how
facial recognition may be tricked using an image of the legitimate user.

% why don't current solution solve these issues
% each paragraph corresponds to one problem
%
% shoulder-surfing
% describe what a shoulder surfing attack is
Shoulder surfing is the practice of spying
on a user for the purpose of
obtaining that user's personal information.
%
When discussing shoulder surfing in this context,
it usually means watching an unaware user 
enter their pin, password, or pattern into
the lock screen.
%
The frequency of smart phone use in public areas
makes shoulder surfers an ever-present threat.
%
Once a malicious party has obtained
the lock screen secret by means of shoulder surfing
they may attempt to steal the phone and bypass the lock screen.
%
\cite{schaub2012password} and \cite{hafiz2008towards} both
cite shoulder surfing as a major concern with 
\cite{schaub2012password} identifying
a (\textgreater$20\%$) success rate
on Android keyboards
for the the shoulder surfing technique
in the reproduction of a password after $3$ attempts.
%
Both works discuss lock screen security mechanisms
and changes which can be made to make them more secure.
However, these changes often resulted in 
decreased usability.
%
In the present state of the lock screen,
studies such as \cite{harbach2014sa} 
which surveyed the lock screen habits of a variety of individuals
have found that
$56.3\%$ of the participants
did not use a lock screen at all.
%
Decreasing the usability of the lock screen even further might
have a worsening effect on overall device security if
it deters individuals from using a lock screen entirely. 

% smudge attacks
% what is a smudge attack
% how does one execute a smudge attack
Smudge attacks attempt to detect
oily residues can expose common touch screen patterns.
%
Attacks can be as simple as using a camera to capture the
these residues for simple image analysis as in
\cite{aviv2010smudge}.
%
This study investigates smudge attacks against
Android's pattern authentication suggesting
an attacker might choose a set of highly likely patterns
from the $389,112$ possible patterns.
%
Another finding is that lock screen patterns could be detectable
even after application use and
incidental clothing contact.
This is significant because 
events which would potentially remove
the oily residues in a real-world scenario
are captured.
%
The question, "is this practical as a method for
phone thieves to recover the lock screen pattern?"
is what is relevant for this paper.
This work \cite{aviv2010smudge}
strongly suggest the answer to
this question should be "yes".

% printable finger prints
Recognizing the limitations of 
traditional authentication mechanisms (pin, password, pattern)
in a mobile environment,
other systems have sought to use a user biometric 
for security.
%
An example of such a system is the finger print scanner 
which has been implemented on many mobile devices today.
The device works by 
asking the user to hold their finger on the scanner for an amount of time.
The scanner reads characteristics of the human finger and
compares the reading against the known user identity.
%
The primary problem with this system is that it may be easily fooled
by through the use of 
fake prints having characteristics similar to those of the
legitimate user.
%
\cite{cao2016hacking} has shown it is possible to fool the
finger print scanner on an Android device into 
false falsely authenticating printed finger print.
The false finger print used
is printed with on AgIC with AgIC ink on AgIC paper,
% say what AgIC is (potentially use same citatioon that was in paper)
a special conductive ink designed for printing circuits
on household printers \cite{AGIC}.
%
Other works by Germany's Chaos Computer Club \cite{CHAOS}
have shown that it is possible to lift a finger print
belonging to the authentic user from the device touch screen.
%
Taken together this suggests that an attacker
could lift a finger print from the device touch screen
belonging to the authentic user and 
print a replica capable of passing 
finger print scanner authentication.

% facial recognition spoofing
Another biometric based authentication method
is facial recognition.
In these systems,
the user presents his or her face to the camera
of their mobile device.
%
The image is then 
analyzed by the system and compared to the existing user identity.
%
% describe an example of this analysis
In \cite{mccool2009parts} this facial analysis
is conducted by 
splitting the image of the face into blocks,
treating each of these blocks as separate sub-images.
Each of these sub-images is then classified using
using a Gaussian Mixture Model (GMM).
%
Importantly this method of analysis relies on
the classification of a static facial image.
%
% replay attacks: what are they, why are they significant
Attacks against facial recognition systems
involve presenting a large number of images from a database
in hopes that one of these images contains
features close enough to those of the legitimate user
to be classified as the user by the system.
%
Other facial recognition systems require movement of
the image being presented
hoping to deter these image attacks.
%
However, similar to the static analysis methods,
these systems may be fooled by
presenting them with a video segment
instead of a static image \cite{hadid2014face}. 
%
\cite{de2013can}
accesses the practically anti-spoofing measures in real word scenarios.
Conclusions from this work suggest
current countermeasures against spoofing attacks,
"lack in generalization and require some improvement".

% describe general issues with these systems
Finger print scanner authentication and
facial recognition are more secure in 
that it is more difficult for an attacker to replicate
the necessary information to pass authentication.
%
However,
these methods are still exploitable in straightforward ways.
%
The issues with these one-shot authentication methods
are tied to the mobile nature of the smart phone.
%
In the case of pin, password, and pattern,
the user must preform these authentications in 
public places.
This exposes the entry process to view of others
enabling shoulder surfing.
%
Additionally,
users must interact with the device using their
fingers as opposed to a mouse and keyboard.
%
Tools other than a user's fingers are quite cumbersome
and decrease usability of the device in the mobile environment.
%
The drawback of the need to use fingers
is the oily residues remaining after the fingers are removed.
This enables smudge attacks to successfully detect
common patterns such as the lock screen pattern.
%
Many authentication schemes which rely on this one-shot
type of authentication will necessarily have this problem.

% look for other security solutions which establish a user
% identity while the device is in use
%
% in other words write about other continuous authentication systems
Continuous authentication schemes exist
that seek to use 
properties of the user interaction
to provide additional security.
%
Schemes such as 
\cite{feng2012continuous}
using touch screen interactions in combination with a glove containing accelerometers, gyroscopes,
and
\cite{frank2013touchalytics}
using $30$ behavioral touch features extracted from touch screen interactions
%
develop a user identity
by utilizing incoming data from multiple device sensors over time.
%TODO there could probabally be ( and probabally should be more here about these works and others)

% say how our system is similar to other continous authentication schemes
Our system is similar to other continuous authentication
systems mentioned above
in that we provide continuous, transparent authentication 
throughout the device use 
after the lock screen.
%
% say how our system is distinct from other continous authentication systems
Our system is distinguished from these systems
by our lack of an additional hardware requirement as in \cite{feng2012continuous}
and our relative simplicity over \cite{frank2013touchalytics}.
These systems also use machine learning in order to classify their users
while
our solution utilizes 
an $n-gram$ Markov model for classification purposes.
%
%TODO write about how other solutions fail in places where ours succeeds
% potentially these other solutions are better in some areas, but 
% at the expense of others, in which ours is better
%TODO

% state the issue in general terms
Having a single authentication mechanism which
verifies the user identity once when the device
state changes from inactive to active has
proven to be an ineffective method for 
delivering device security.
%
The need for a more comprehensive security solution more is evident.
%
We propose that a system which uses
a biometric of user touch screen interaction to
establish a user, device identity would 
be more comprehensive compared to solutions 
which are currently available.

% how does our solution solve these issues
% (assume the claims are correct, prove claims later)
% make a case for why, how our system solves current issues
Our solution is more comprehensive due to
both 
the biometric nature of the data used and
the operation throughout the use of the device.
%
% talk about how these two ^ things make our solution an improvement
% improvement over one-shot authentication
The biometric nature of the data makes the
identity of the user much more difficult to reproduce while
%
the operation of the system throughout the lifetime of 
a user's interaction with the device allows
continuous testing of the user's identity.
%
% the two together are more secure than either on their own
The need to reproduce a user biometric throughout the entire attack
against a device further improves the security offered by our solution.
%
% improvement over other continuous authentication ( in general )
Over other continuous authentication systems,
we provide improvements 
in usability over \cite{feng2012continuous}
and in simplicity over \cite{frank2013touchalytics}.
