% here is the problem
% it is an interesting problem
% NOTE: remember to introduce problems using examples
% 1: present example
% 2: describe problems in example
% 3: present the general case of problems
\section{The Problem}
\label{the_problem}
% what are the issues ( restate from introduction )
Current protection mechanisms do not adequately protect
data contained on mobile devices.
The current scheme provides a single line of defense,
the lock screen.
If this line of defense has been compromised
by an attacker,
all applications and data on the device become available.
%
This scheme might be considered successful 
if the lock screen offered a high level of device security.
Unfortunately, there are exploits for many common lock screen 
authentication mechanisms 
allowing an attacker to bypass the protection.
%
Common protection mechanisms include
pin, password, pattern, finger print scanners, and facial recognition.
%
The first three can be overcome successfully with 
shoulder-surfing \cite{schaub2012password} and smudge attacks \cite{aviv2010smudge}.
Finger print scanners and facial recognition require a different approach
where the relevant features of the legitimate user are spoofed
in order to pass authentication.
%
A system is created in \cite{cao2016hacking} 
to easily and quickly print finger prints
capable of passing finger print scanner authentication.
%
Similarly \cite{de2013can} describes how
facial recognition may be tricked using an image of the legitimate user.

% why don't current solution solve these issues
% each paragraph corresponds to one problem
%
%TODO shoulder-surfing
% describe what a shoulder surfing attack is
%Shoulder surfing is a major concern

%TODO smudge attacks
%TODO printable finger prints
%TODO facial recognition spoofing

% discuss the issues with lock screen in more detail
%TODO integrate each of these problems ( talk about all of them in further detail)
%%%%%%%%%%%%%%%%%%%
\cite{schaub2012password} discusses the vulnerability of
various touch screen input methods to shoulder-surfing attacks
while
\cite{hafiz2008towards} measured the susceptibility of 
users to shoulder surfing attacks observing
attackers having a ($\textgreater20\%$) success rate
on Android keyboards
in the reproduction of a password after $3$ attempts.
%
Smudge attacks where,
oily residues are used to detect common touch screen patters,
also represent a significant security risk
for secrets input via the touch screen.
\cite{aviv2010smudge} explores the feasibility of
such attacks finding in one experiment
the lock screen pattern was partially identifiable $92\%$ of the time
and full identifiable $68\%$ of the time.

% use some other examples of lock screens
These vulnerabilities have led to the development of 
lock screens which 
use a user biometric in an attempt to increase security.
Facial recognition and finger print scanners have
been employed as lock screen mechanisms.
% finger print scanners and facial recognition
These methods insufficient to protect the device from attack.
%
% talk about how finger print scanners and facial recognition may be exploited
\cite{cao2016hacking} demonstrates a method 
to exploit the finger print scanner on an android device
using printed fingerprints.
This method is simple and fast as long as a print of the
authentic user's finger can be obtained.
%
Other works by Germany's Chaos Computer Club \cite{CHAOS}
have show that it is possible to lift a finger print 
belonging to the authentic user from the device touch screen
and use it to pass the finger print scanner authentication on Apple's Iphone $4S$ and $5$. 
%
Attacks against facial recognition schemes involve
spoofing the facial features of the authentic user
in order to pass facial recognition authentication.
\cite{de2013can}
accesses the practically anti-spoofing measures in real word scenarios
observing 
low generalization of
and possible database bias in existing schemes.

In many situations,
an attacker may not even need to pass the lock screen authentication.
% mobile devices can change hands while the authentication state is still active
Imagine a situation where the authentic user of a device preforms the authentication,
unlocking the device.
The device then falls into the hands of an attacker.
This attacker can not only exploit the authenticated state of the device,
but may use the authenticated state to disable the lock screen entirely;
Thus allowing permanent access to all applications on the device.
%
% many people neglect to use a lock screen at all
% demographic
\cite{harbach2014sa} surveyed the lock screen behavior of $320$
individuals ages $18 - 67$ with a median age of $31$.
Of these individuals, $39.6\%$  rated themselves as having
a very high level of IT expertise.
% findings
The results of this survey indicate $56.3\%$ of the participants
did not use a lock screen at all.
%%%%%%%%%%%%%%%%%%%%

%TODO look for other security solutions which establish a user
%TODO identity while the device is in use

% state the issue in general terms
Having a single authentication mechanism which
verifies the user identity once when the device
state changes from inactive to active has
proven to be an ineffective method for 
delivering device security.
%
The need for a more comprehensive security solution more is evident.
%
We propose that a system which uses
a biometric of user touch screen interaction to
establish a user, device identity would 
be more comprehensive compared to solutions 
which are currently available.

% how does our solution solve these issues
% (assume the claims are correct, prove claims later)
% make a case for why, how our system solves current issues
Our solution is more comprehensive due to
both 
the biometric nature of the data used and
the operation throughout the use of the device.
%
% talk about how these two ^ things make our solution an improvement
% improvement over one-shot authentication
%TODO
%
% improvement over other continuous authentication
%TODO