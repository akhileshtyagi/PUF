% outlined using the Heilmeier questions
\section{Introduction}
% What are you trying to do? Articulate your objectives using absolutely no jargon.  What is the problem?  Why is it hard?
Malicious parties attempt to gain access to data of their victims. 
Often times these attackers go through the trouble of
preforming these attacks because
gaining access to the data might be quite lucrative.
Take the example of 
%TODO cite some sort of attack on mobile device data which results in 
  %the attacker stealing money.

  % many applications make assumptions about the device.
  % that once a user has authenticated,
  % possession of the device is sufficient to
  % access the services

% BAD EXAMPLE
% a user who uses mobile applications for monetary transactions.
% This application uses a user name and password to 
% establish the identity of a user.
% If an attacker is able to acquire this information,
% the user name and password,
% then the attacker will have the ability use money
% kept within the legitimate user's account.
% This is not acceptable.

% How is it done today, and what are the limits of current practice?
Many current practices keep mobile devices secure
utilizing some nature of lock screen.
These protection mechanisms require
that the user preform some action 
when the mobile device state moves from inactive to active.
This one-time authentication allows access to a large amount 
applications, tools, and data on the device.
Some of this data could be potentially
damaging if compromised by a malicious party.

One issue with this nature of user verification %TODO
There are many ways an attacker might trick
this one-time authentication 
hence gaining access to resources on the device.
%TODO cite some attacks like "look over shoulder attacks"
%TODO

This paper focuses on an idea for creating
added security for data stored on mobile devices.
Often times there exists a trade off between
security and convenience from the user's perspective.
This can be seen in solutions such as
%TODO cite a mobile device security solution which has poor convenience to the user
where
%TODO describe this solution and how it represents a tradeoff between security and convenience
The solution presented in this paper
provides additional security while requiring
no additional actions from the user.
This is accomplished by
recording touch screen interactions
in the background while the
user interacts as they normally would
with the applications on their mobile device.
The most resent interactions are then
tested against previous actions
to determine if the user's
pattern has changed.

We provide a step forward, 
enhancing the ability of mobile devices to
recognize when the device may have been compromised.
Compromised here meaning the device is in
the physical possession of an illegitimate user.
This system provides innovations in the following areas:
\begin{itemize}
\item We show that touch screen interactions may be used in order to
  distinguish a legitimate user of a mobile device apart from
  illegitimate users.
  Specifically, section
  \ref{the_solution} % give forward reference
  shows how a combination of 
  a user's pressure when touching the screen and 
  the location on the screen being touched
  may be used to develop a model of that user's behavior.
\item We establish the behavior of a user on a device
  is unique to both the device and the user.
  Section 
  \ref{the_details} %TODO give forward reference
  establishes that a difference in either device or user
  may be detected by our implementation.
\item There is no convenience cost assessed upon the user.
  All security improvements are accomplished without requiring any additional
  actions from the user.
  The specifications of the implementation used to accomplish 
  this transparency are discussed in Section
  \ref{the_solution} % give forward reference
\item The performance of this system is sufficient to make it practical.
  %TODO elaborate on what types of applications this is practical for,
  % or on the definition of practicality
  A performance comparison is presented in Section
  \ref{the_details} %TODO give forward reference
  which demonstrates the running time of the system on 
  a Nexus 7 tablet.
  % Section ? discusses the memory requirements of the system,
  % showing that they are ?. 
%TODO memory requirements of the system?
\item We improve upon the ideas discussed in ? %TODO reference mellon sensec paper
  utilizing similar concepts for modeling interactions. 
  We extend beyond this work,
  modeling soft keyboard interactions
  to describe user behavior over time.
  Section
  \ref{related_work} % forward reference related work
  provides further discussion of ?%TODO reference mellon sensec paper 
  with emphasis on similarities and differences
  between work described throughout this paper.
\item This work solves the problem of x %TODO describe problem
  by ensuring data security in situation y. %TODO describe problematic situation
  Other solutions fail to consider this problem
  as discussed in Section
  \ref{the_problem}.
\end{itemize}

% What's new in your approach and why do you think it will be successful?
In deciding if a user is legitimate,
it is useful to define three categories:
something the user knows,
something the user has, and
something the user is.
Traditional mobile authentication schemes,
such as a lock screen,
utilize only something a user knows.
Interactions between a user
and the touchscreen of a mobile device
are rich with information.
Current solutions suffer from underutilization 
of this information; 
they discard much of the content of these
interactions in favor using
the location of the interactions exclusively.
%
Our system utilizes 
pressure,
time, and
location 
capturing the currently under-utilized potential 
of these interactions to expose patterns unique to a user.

% describe the modeling scheme
We use these properties to construct
a model of how the user interacts with a mobile device.
This model takes as input a sequence of touches preformed
by the user.
From this sequence
probabilities are developed
which represent the likelihood of a given touch screen interaction
based on the properties of a number of previous interactions.
A type Markov model which uses $n$ previous states
in computing next state probabilities
is used for the purpose of developing these probabilities.
This model is explained more throughly in
Section \ref{the_solution}. % give forward reference

% Who cares?

% If you're successful, what difference will it make?   What impact will success have?  How will it be measured?

% What are the risks and the payoffs?

% How much will it cost?

% How long will it take?
% What are the midterm and final "exams" to check for success?  How will progress be measured?