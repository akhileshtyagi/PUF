Features could become a sequence of touches.
Different n-gram sizes each.

try using raw pressure sequences as features.
$1000$ features for a $1000$ sequence.
$1000$ features predicts a class.

Discuss why it is difficult to give a raw data dump to machine learning.
It is the sequences that matter.
Not the individual points.


\subsection{Classification on Raw Data}
The classifier is given raw touchscreen data.
For each user interaction the classifer is given
one row in the form
{\tt "key", "pressure", "user"}.

The classifier is then asked if it can correctly classify
a single observation of
{\tt "key", "pressure", "user"}.

While the misclassification rate might be large with
only one observation,
using multiple observations for classification reduce
misclassifications significantly.
%For instance if one observation can
%classify a user with $20\%$ accuracy

\begin{verbatim}
Resampling: Cross-Validated (10 fold, repeated 3 times) 
Summary of sample sizes: 11700, 11700, 11700, 11700, 11700, 11700, ... 
Resampling results across tuning parameters:

  cost  Accuracy   Kappa     
  0.25  0.1683333  0.09902778
  0.50  0.1683333  0.09902778
  1.00  0.1681026  0.09877778

Accuracy was used to select the optimal model using  the largest value.
The final value used for the model was cost = 0.25. 
\end{verabatim}

\subsection{Classification on Chain Characteristics}
\subsubsection{Transition Probabilities}
%TODO this is like the chan data that i'm reading in now

\subsubsection{State Probabilities}
%TODO how might this be done
